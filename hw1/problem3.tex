\documentclass[11pt]{article}
\usepackage{subfigure}
\usepackage{color}
\usepackage{url}
\usepackage{graphicx}
\usepackage{fullpage}
\usepackage[english]{babel}
\usepackage{amssymb}
\usepackage{amsmath}
\usepackage{fancyhdr}
\usepackage{hyperref}
\usepackage{algorithmic}
\usepackage{algorithm}
\usepackage{enumerate}
\usepackage{mdframed}
\usepackage{mathrsfs}


\begin{document}
\begin{center}
%---------------------------------------------------------------------------------------
%---------------------------------Header------------------------------------------------
%---------------------------------------------------------------------------------------

\framebox{\parbox{6.5in}{
{\bf{STATS 315B: Data Mining, Spring 2016}}\\
{\bf Homework 1, Due 4/28/2016}\\
{\bf Completed by: Henry Neeb, Christopher Kurrus, Tyler Chase, and Yash Vyas}
}}
\ \\
\end{center}

\section*{Problem 3}

$$Target Function: F^* = argmin_FR(F)$$
$$Risk Function: R(F) = E_{XY}L(Y, F(\underline{X})$$
$$Loss Criterion: L(Y,F(\underline{X}))$$

\vspace{5 mm}
\noindent
Y is the actual output and \underline{X} is a vector of predictors. Even though 
The target function is the optimal function to minimize prediction risk, it is 
not always an accurate function for prediction. It could be that there is very 
little signal between the predictors and the response. That is, there is very 
low signal to noise.

\vspace{5 mm}
\noindent
This could arise when your predictors do not do well at predicting your 
response. This could arise over time as well when your problem exhibits 
concept drift.

\end{document}