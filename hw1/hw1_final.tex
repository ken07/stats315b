\documentclass[11pt]{article}
\usepackage{subfigure}
\usepackage{color}
\usepackage{url}
\usepackage{graphicx}
\usepackage{fullpage}
\usepackage[english]{babel}
\usepackage{amssymb}
\usepackage{amsmath}
\usepackage{fancyhdr}
\usepackage{hyperref}
\usepackage{algorithmic}
\usepackage{algorithm}
\usepackage{enumerate}
\usepackage{mdframed}
\usepackage{mathrsfs}

\DeclareMathOperator*{\argmin}{argmin}
\newcommand*{\argminl}{\argmin\limits}

\begin{document}

%-------------------------------------------------------------------------------
%---------------------------------Problem 3-------------------------------------
%-------------------------------------------------------------------------------
\begin{center}
\framebox{\parbox{6.5in}{
{\bf{STATS 315B: Data Mining, Spring 2016}}\\
{\bf Homework 1, Due 4/28/2016}\\
{\bf Completed by: Henry Neeb, Christopher Kurrus, Tyler Chase, and Yash Vyas}
}}
\ \\
\end{center}

\section*{Problem 3}

$$Target Function: F^* = argmin_FR(F)$$
$$Risk Function: R(F) = E_{XY}L(Y, F(\underline{X})$$
$$Loss Criterion: L(Y,F(\underline{X}))$$

\vspace{5 mm}
\noindent
Y is the actual output and \underline{X} is a vector of predictors. Even though 
The target function is the optimal function to minimize prediction risk, it is 
not always an accurate function for prediction. It could be that there is very 
little signal between the predictors and the response. That is, there is very 
low signal to noise.

\vspace{5 mm}
\noindent
This could arise when your predictors do not do well at predicting your 
response. This could arise over time as well when your problem exhibits 
concept drift.

%-------------------------------------------------------------------------------
%---------------------------------Problem 4-------------------------------------
%-------------------------------------------------------------------------------
\newpage
\begin{center}
\framebox{\parbox{6.5in}{
{\bf{STATS 315B: Data Mining, Spring 2016}}\\
{\bf Homework 1, Due 4/28/2016}\\
{\bf Completed by: Henry Neeb, Christopher Kurrus, Tyler Chase, and Yash Vyas}
}}
\ \\
\end{center}

\section*{Problem 4}

$$Population Risk: R(F) = E_{xy}L(Y,F(\underline{X}))$$
$$ Empirical Risk: R(F) = \frac{1}{N}\sum_{i=1}^NL(Y_i, F(\underline{X}_i))$$ 

\vspace{5 mm}
\noindent
The empirical risk does not average over all possible X distributions so 
$F(\underline{X})$ is thus more likely to overfit a data set 
$\{\underline{X}_i, Y_i\}_1^N$, if we use the empirical risk to find optimal F 
($F^* = argmin_FR(F)$). If we approximate Y by 
$Y=F^*(\underline{X}) + \epsilon$ we in other words run the risk of our model 
$F^*$ following the specific random noise $\epsilon$ in our training set when 
using empirical risk as opposed to population risk. It is possible for the 
empirical risk to be zero while the population risk is very large if we 
consider a function that doesn't behave the appropriate way for the data we are 
considering, but passes through all of the  data points. 

\vspace{5 mm}
\noindent
Empirical risk does not perform well as a surrogate to prediction risk when 
you have poorly sampled your training data. That is, you did not sample your 
training data from the same distribution as your population data.

\vspace{5 mm}
\noindent
Empirical risk can be expected to be a good surrogate when you choose an 
appropriate function class $\mathscr{F}$ (where 
$\hat{F}(\underline{X}) \in \mathscr{F}$) that will prevent overfitting 
(increasing variance) on the training set $\{\underline{X}_i, Y_i\}_1^N$. 
Overfitting can also be mitigated by collecting or using much more data such 
that the data is sampled from the population distribution. Also, using cross 
validation with testing and training data will help reduce over fitting.

%-------------------------------------------------------------------------------
%---------------------------------Problem 5-------------------------------------
%-------------------------------------------------------------------------------
\newpage
\begin{center}
\framebox{\parbox{6.5in}{
{\bf{STATS 315B: Data Mining, Spring 2016}}\\
{\bf Homework 1, Due 4/28/2016}\\
{\bf Completed by: Henry Neeb, Christopher Kurrus, Tyler Chase, and Yash Vyas}
}}
\ \\
\end{center}

\section*{Problem 5}

\vspace{5 mm} 
\noindent 
Searching over the entire space of function is both computationally infeasible 
(you cannot possibly hope to search over an infinite space) and it would most 
likely yield suboptimal results. We define our target function as follows:

\begin{gather*}
F^{*} = \argminl_{F} R(F) = \argminl_{F} E_{XY}[L(y, F(x))]
\end{gather*}

\noindent
Where:

\begin{itemize}
\item $L(y, F(x)) \ge 0$
\item $y = F(x) \rightarrow L(y, F(x)) = 0$
\end{itemize}

\noindent
In practice, we typically estimate the target function by minimizing the
empirical risk over the training data as a surrogate to minimizing the
prediction risk. So if we develop our model such that $F(x) = y$ for our
training data, we will have found a model such that its total training loss is
$0$ and our empirical risk on the training data is absolutely minimized. There 
is not only just one such model that will fit your training data exactly, but
infinitely many models. Your search for the function that minimizes empirical
risk will then conclude with a set of infinitely many functions to choose from
as your best approximation for the target function.

\vspace{5 mm}
\noindent
Not only will it be infeasible to distinguish between these infinitely many
functions, but all of the functions returned will most likely be nowhere near a
good approximation to the target function. This is due to the fact your training
data is random itself and has noise. When you fit the training data exactly, you
are almost guaranteed to be over-fitting since you will be modeling the noise
exhibited in your particular training data. When you start fitting the noise,
your function loses its ability to predict on new data because that new data may
have the same signal as the old data, but it has different noise. As such, your 
model will exhibit prediction error, and most likely will have more prediction 
error than if you chose a model by restriction the function class to search 
over.

%-------------------------------------------------------------------------------
%---------------------------------Problem 6-------------------------------------
%-------------------------------------------------------------------------------
\newpage
\begin{center}
\framebox{\parbox{6.5in}{
{\bf{STATS 315B: Data Mining, Spring 2016}}\\
{\bf Homework 1, Due 4/28/2016}\\
{\bf Completed by: Henry Neeb, Christopher Kurrus, Tyler Chase, and Yash Vyas}
}}
\ \\
\end{center}

\section*{Problem 6}

\vspace{5 mm}
\noindent
Bias and variance are two sources of error that occur when searching for the 
target function. {\bf Bias} is the error associated with your expected model 
not being able to predict the target model. Bias occurs when you make 
assumptions about your target function and restrict your search to a specific 
function class, which almost certainly will not contain the target function. 
Since the target function is outside of the class you are searching over, you 
can never expect to predict the true target function.

\vspace{5 mm}
\noindent
{\bf Variance} is the error associated with how different a model you are 
expected to fit is within a chosen function class. Typically, the more 
restrictive the search space, the less the variance because the possibilities of 
models chosen given your training data is that much less.

\vspace{5 mm}
\noindent
The {\bf bias-variance tradeoff} comes from the tradeoff of trying to be broad 
and flexible in your search of the target function in order to actually be more 
likely to recover it in your search (reducing bias), but not being too 
unrestrictive in your search as to have a volatile model (that is a model that 
will change dramatically with small changes in the training data). That is, 
reducing bias (expanding your search region for the target function), is often 
at odds with reducing variance (restricting your search region to have a more 
stable model). You will often never be able to eliminate both, and when trying 
to reduce your overall error, you will need to make concessions to both types 
of error when evaluating a model.

%-------------------------------------------------------------------------------
%---------------------------------Problem 8-------------------------------------
%-------------------------------------------------------------------------------
\newpage
\begin{center}
\framebox{\parbox{6.5in}{
{\bf{STATS 315B: Data Mining, Spring 2016}}\\
{\bf Homework 1, Due 4/28/2016}\\
{\bf Completed by: Henry Neeb, Christopher Kurrus, Tyler Chase, and Yash Vyas}
}}
\ \\
\end{center}

\section*{Problem 8}

\vspace{5 mm}
\noindent
It would be incorrect to use surrogate splits to predict the outcome variable, 
as opposed to using the primary split, because it will not result in an optimal 
(best) split.  We know that the surrogate split will not be optimal for the 
points where we have non-missing values, as otherwise it would be the primary 
split, and we can't apply the primary split to the missing values.  This means 
that our best split will be applying the primary split to the non-missing 
values and classifying missing values to the appropriate daughter region by 
using the surrogate variable.

%-------------------------------------------------------------------------------
%---------------------------------Problem 9-------------------------------------
%-------------------------------------------------------------------------------
\newpage
\begin{center}
\framebox{\parbox{6.5in}{
{\bf{STATS 315B: Data Mining, Spring 2016}}\\
{\bf Homework 1, Due 4/28/2016}\\
{\bf Completed by: Henry Neeb, Christopher Kurrus, Tyler Chase, and Yash Vyas}
}}
\ \\
\end{center}

\section*{Problem 9}

\vspace{5 mm}
\noindent
To show that the minimizing values of $\hat{c}_{m}$ are given by 
$\sum\limits_{i=0}^N (y_{i} I(x_{i} \in R_{m})) / 
\sum\limits_{i=0}^N (I(x_{i} \in R_{m}))$ 
we will expand upon our given functions.

\vspace{3 mm}
\noindent
We know our criterion is $\sum\limits_{i=1}^N (y_{i} - F(x_{i}))^2$ and we have 
$F(x) = \sum\limits_{m=1}^M (c_{m}I(x \in R_{m}))$

\vspace{3 mm}
\noindent
Proceeding, we can expand our criterion into 
$\sum\limits_{i=1}^N (y_{i} - \sum\limits_{m=1}^M (c_{m}I(x_{i} \in R_{m})))^2$

\vspace{3 mm}
\noindent
Now our goal is to find the minimizing value 
$min_{c_{m}}[\sum\limits_{i=1}^N (y_{i} - 
\sum\limits_{m=1}^M (c_{m}I(x_{i} \in R_{m})))^2]$ 

\vspace{3 mm}
\noindent
We will find our minimizing values by taking the partial derivative on $c_{m}$ 
for an arbitrary region m.

\vspace{3 mm}
\noindent
$\frac{d}{dc_{m}}[\sum\limits_{i=1}^N (y_{i} - 
\sum\limits_{m=1}^M (c_{m}I(x_{i} \in R_{m})))^2] = 
2\sum\limits_{i=1}^N [(y_{i} - \sum\limits_{m=1}^M (c_{m}I(x_{i} \in R_{m})))
(I(x_{i} \in R_{m}))]$ 

\vspace{3 mm}
\noindent
Now that we have done our differentiation we simplify our terms by distributing 
the indicator term that is evaluating on our chosen m to the remaining terms 
that are non-chosen values of m. For our $y_{i}$ term the indicator stays, but 
this isn't true for all values in the summation for 
$(c_{m}I(x_{i} \in R_{m}))$.  We know that the regions are all disjoint, so 
every cross term between our indicator for the chosen region, and indicators 
for every other region will evaluate to zero.  This leaves only the 
$(c_{m}I(x_{i} \in R_{m}))I(x_{i} \in R_{m})$ term, but the indicator squared 
is merely the indicator, so we end up with the statement below.

\vspace{3 mm}
\noindent
$\sum\limits_{i=1}^N [y_{i}I(x_{i} \in R_{m}) - c_{m}I(x_{i} \in R_{m})] = 0$ 

\vspace{3 mm}
\noindent
Splitting the summation we have 
$\sum\limits_{i=1}^N (y_{i}I(x_{i} \in R_{m}) - 
c_{m}\sum\limits_{i=1}^N (I(x_{i} \in R_{m}) = 0$

\vspace{3 mm}
\noindent
So simplifying, 
${c}_{m} = \frac{\sum\limits_{i=1}^N (y_{i} I(x_{i} \in R_{m}))}
{\sum\limits_{i=1}^N (I(x_{i} \in R_{m}))}$ 
as desired.

%-------------------------------------------------------------------------------
%---------------------------------Problem 10------------------------------------
%-------------------------------------------------------------------------------
\newpage
\begin{center}
\framebox{\parbox{6.5in}{
{\bf{STATS 315B: Data Mining, Spring 2016}}\\
{\bf Homework 1, Due 4/28/2016}\\
{\bf Completed by: Henry Neeb, Christopher Kurrus, Tyler Chase, and Yash Vyas}
}}
\ \\
\end{center}

\section*{Problem 10}
\setlength{\parindent}{0pt}

From the lecture notes the improvement in square error risk when one of the 
regions $R_m$ is split into daughter regions $R_m \to R_l \cup R_r$

$\underset{x}{\operatorname{argmax}}$

\begin{equation}
m^* = \underset{1 \leq m \leq M}{\operatorname{argmax}} 
\sum_{\underline{X}_i \in R_m} \Bigg\{\Big[y_i-\bar{y}_m\Big]^2-
\Big[y_i - \bar{y}_m^{(l)}I(\underline{X}_i \in R_m^{(l)})-
\bar{y}_m^{(r)}I(\underline{X}_i \in R_m^{(r)})\Big]^2\Bigg\}
\end{equation}

We now expand the squared terms. 

\begin{equation}
\begin{split}
m^* = \underset{1 \leq m \leq M}{\operatorname{argmax}} 
\sum_{\underline{X}_i \in R_m} \Bigg\{(y_i^2-y_i^2) -2y_i\bar{y}_m + \bar{y}_m^2 
+2y_i\bar{y}_m^{(l)}I(X_i \in R_m^{(l)}) 
+2y_i\bar{y}_m^{(r)}I(\underline{X}_i \in R_m^{(r)}) \\
- 2y_m^{(l)}I(\underline{X}_i \in R_m^{(l)})\bar{y}_m^{(r)}I(\underline{X}_i 
\in R_m^{(r)})-[\bar{y}_m^{(l)}I(\underline{X}_i 
\in R_m^{(l)})]^2-[\bar{y}_m^{(r)}I(\underline{X}_i \in R_m^{(r)})]^2\Bigg\}
\end{split}
\end{equation}

Cancel the two $y_i^2$ terms. Since $R_m^{(l)} \cap R_m^{(r)} = 0$ the term  
$2y_m^{(l)}I(\underline{X}_i 
\in R_m^{(l)})\bar{y}_m^{(r)}I(\underline{X}_i \in R_m^{(r)})$ 
cancels. Also note that $I(\underline{X}_i \in R)^2 = I(\underline{X}_i \in R)$.

\begin{equation}
\begin{split}
m^* = \underset{1 \leq m \leq M}{\operatorname{argmax}} 
\sum_{\underline{X}_i \in R_m} \Bigg\{-2y_i\bar{y}_m + \bar{y}_m^2 + 
2y_i\bar{y}_m^{(l)}I(\underline{X}_i \in R_m^{(l)}) + 
2y_i\bar{y}_m^{(r)}I(\underline{X}_i \in R_m^{(r)})\\ 
- [\bar{y}_m^{(l)}]^2I(\underline{X}_i \in R_m^{(l)}) - 
[\bar{y}_m^{(r)}]^2I(\underline{X}_i \in R_m^{(r)})\Bigg\}
\end{split}
\end{equation}

Next we use the fact that 
$\sum_{\underline{X_i} \in R_m} y_i = N_m\frac{1}{N_m} 
\sum_{\underline{X}_i \in R_m}y_i = N_m \bar{y}_m$

\begin{equation}
\begin{split}
m^* = \underset{1 \leq m \leq M}{\operatorname{argmax}} 
\Bigg\{-2N_m\bar{y}_m^2 +N_m\bar{y}_m^2 + 2N_m^{(l)}[\bar{y}_m^{(l)}]^2 +
 2N_m^{(r)}[\bar{y}_m^{(r)}]^2  -N_m^{(l)}[\bar{y}_m^{(l)}]^2 -N_m^{(r)}
 [\bar{y}_m^{(r)}]^2\Bigg\}
\end{split}
\end{equation}

Cancel the like terms. 

\begin {equation}
\begin{split}
m^* = \underset{1 \leq m \leq M}{\operatorname{argmax}} 
\Bigg\{-N_m[\bar{y}_m]^2 + N_m^{(l)}[\bar{y}_m^{(l)}]^2 
+N_m^{(r)}[\bar{y}_m^{(r)}]^2\Bigg\}
\end{split}
\end{equation}

We next use the fact that $N_m\bar{y}_m = N_m^{(l)}\bar{y}_m^{(l)} + 
N_m^{(r)}\bar{y}_m^{(r)}$ proven below.

$$ \sum_{\underline{X}_i \in R_m} y_i = \sum_{\underline{X}_i 
\in R_m^{(l)}}y_i + \sum_{\underline{X}_i \in R_m^{(r)}} y_i 
\Rightarrow N_m \frac{1}{N_m} \sum_{\underline{X}_i 
\in R_m}y_i = N_m^{(l)}\frac{1}{N_m^{(l)}} 
\sum_{\underline{X}_i \in R_m^{(l)}}y_i + N_m^{(r)}\frac{1}
{N_m^{(r)}}\sum_{X_i \in R_m^{(r)}}y_i $$

\begin {equation}
m^* = \underset{1 \leq m \leq M}{\operatorname{argmax}} 
\Bigg\{ -N_m ( \frac{N_m^{(l)}\bar{y}_m^{(l)} + 
N_m^{(r)}\bar{y}_m^{(r)}}{N_m})^2 + N_m^{(l)}[\bar{y}_m^{(l)}]^2 + 
N_m^{(r)}[\bar{y}_m^{(r)}]^2\Bigg\}
\end{equation}

Make a common denominator and use $N_m = N_m^{(l)} + N_m^{(r)}$.

\begin {equation}
\begin{split}
m^* = \underset{1 \leq m \leq M}{\operatorname{argmax}} 
\Bigg\{\frac{-[N_m^{(l)}\bar{y}_m^{(l)}]^2 - 
[N_m^{(r)}\bar{y}_m^{(r)}]^2-2N_m^{(l)}\bar{y}_m^{(l)}
N_m^{(r)}\bar{y}_m^{(r)}}{N_m}\\ 
+ \frac{(N_m^{(l)}+ N_m^{(r)})N_m^{(l)}[\bar{y}_m^{(l)}]^2 + 
(N_m^{(l)}+N_m^{(r)})N_m^{(r)}[\bar{y}_m^{(r)}]^2}{N_m}\Bigg\}
\end{split}
\end{equation}

Cancel the like terms and pull out a common factor in all terms. 

\begin{equation}
m^* = \underset{1 \leq m \leq M}{\operatorname{argmax}} 
\Bigg\{\frac{N_m^{(l)}N_m^{(r)}}{N_m}\Big([\bar{y}_m^{(l)}]^2-
2\bar{y}_m^{(l)}\bar{y}_m^{(r)} + [\bar{y}_m^{(r)}]^2\Big)\Bigg \}
\end{equation}

\begin{equation}
m^* = \underset{1 \leq m \leq M}{\operatorname{argmax}} 
\Bigg\{\frac{N_m^{(l)}N_m^{(r)}}{N_m}\Big(\bar{y}_m^{(l)} - 
\bar{y}_m^{(r)}\Big)^2\Bigg \}
\end{equation}

%-------------------------------------------------------------------------------
%---------------------------------Problem 11------------------------------------
%-------------------------------------------------------------------------------
\newpage
\begin{center}
\framebox{\parbox{6.5in}{
{\bf{STATS 315B: Data Mining, Spring 2016}}\\
{\bf Homework 1, Due 4/28/2016}\\
{\bf Completed by: Henry Neeb, Christopher Kurrus, Tyler Chase, and Yash Vyas}
}}
\ \\
\end{center}

\section*{Problem 11}
\setlength{\parindent}{0pt}

The goal is to derive an updating formula for calculating the change in 
improvement  in prediction risk as a result of a split when the split is 
modified by one observation changing sides. 

In problem 10 we showed that the improvement in prediction risk as a result of 
a split is 

$$ \frac{n_ln_r}{n}\big(\bar{y}_l - \bar{y}_r\big)^2$$

Assume the convention that an observation moving from the left to right 
daughter is positive and label the observation as $(\underline{x}_k, y_k)$

\begin{equation}
n_l \rightarrow n_l - 1
\end{equation}

\begin{equation}
n_r \rightarrow n_r + 1
\end{equation}

\begin{equation}
\bar{y}_l = \frac{1}{n_l}\sum_{\underline{x}_i \in R_m^{(l)}}y_i
\end{equation}

\begin{equation}
\bar{y}_r = \frac{1}{n_r}\sum_{\underline{x}_i \in R_m^{(r)}}y_i
\end{equation}

\begin{equation}
\sum_{ \underline{x}_i \in R_m^{(l)}} y_i \rightarrow \bigg( 
\sum_{ \underline{x}_i \in R_m^{(l)}}y_i \bigg) - y_k
\end{equation}

\begin{equation}
\sum_{ \underline{x}_i \in R_m^{(r)}} y_i \rightarrow \bigg( 
\sum_{ \underline{x}_i \in R_m^{(r)}}y_i \bigg) - y_k
\end{equation}

$$(1), (3), (5) \Rightarrow \bar{y}_l = \frac{1}{n_l}\sum_{\underline{x}_i 
\in R_m^{(l)}}y_i \rightarrow \frac{1}{n_l-1}\Bigg[\bigg(\sum_{\underline{x}_i 
\in R_m^{(l)}}y_i\bigg)-y_k\Bigg] = \frac{n_l}{n_l-1}
\Bigg[\bigg(\frac{1}{n_l}\sum_{\underline{x}_i \in R_m^{(l)}}y_i\bigg) - 
\frac{y_k}{n_l}\Bigg]$$

\begin{equation}
\bar{y}_l \rightarrow \frac{n_l}{n_l-1}\bar{y}_l - \frac{y_k}{n_l-1}
\end{equation}

$$(2), (4), (6) \Rightarrow \bar{y}_r = \frac{1}{n_r}\sum_{\underline{x}_i 
\in R_m^{(r)}}y_i \rightarrow \frac{1}{n_r+1}\Bigg[\bigg(\sum_{\underline{x}_i 
\in R_m^{(r)}}y_i\bigg)+y_k\Bigg] = \frac{n_r}{n_r+1}\Bigg[\bigg(\frac{1}{n_r}
\sum_{\underline{x}_i \in R_m^{(r)}}y_i\bigg) + \frac{y_k}{n_r}\Bigg]$$

\begin{equation}
\bar{y}_r \rightarrow \frac{n_r}{n_r+1}\bar{y}_r - \frac{y_k}{n_r+1}
\end{equation}

Below is the improvement in making a split as a function of the original 
variables, with one point moved from the left daughter region to the right 
daughter region. 

$$(1), (2), (7), (8) \Rightarrow \frac{n_ln_r}{n} \big(\bar{y}_l - 
\bar{y}_r\big)^2 \rightarrow \frac{(n_l-1)(n_r+1)}{n}
\bigg(\frac{n_l}{n_l+1}\bar{y}_l - \frac{y_k}{n_l-1} - 
\frac{n_r}{n_r+1}\bar{y}_r - \frac{y_k}{n_r + 1}\bigg)^2$$

Below is the improvement as one point is moved from the left daughter region to 
the right daughter region. 

$$\frac{n_ln_r}{n} \big(\bar{y}_l - \bar{y}_r\big)^2 - 
\frac{(n_l-1)(n_r+1)}{n}\bigg(\frac{n_l}{n_l+1}\bar{y}_l - \frac{y_k}{n_l-1} - 
\frac{n_r}{n_r+1}\bar{y}_r - \frac{y_k}{n_r + 1}\bigg)^2$$
%-------------------------------------------------------------------------------
%---------------------------------Problem 12------------------------------------
%-------------------------------------------------------------------------------
\newpage
\begin{center}
\framebox{\parbox{6.5in}{
{\bf{STATS 315B: Data Mining, Spring 2016}}\\
{\bf Homework 1, Due 4/28/2016}\\
{\bf Completed by: Henry Neeb, Christopher Kurrus, Tyler Chase, and Yash Vyas}
}}
\ \\
\end{center}

\section*{Problem 12}

\vspace{5 mm}
\noindent
{\bf Treat Node as Terminal:} Treating the node as terminal means we predict on 
the majority at the node with the missing value. The following are reasons and 
situations when this strategy is ok:

\begin{itemize}
\item Surrogate might be a bad predictor. Weak correlation might split 
erroneously.
\item If you start with homogeneous data (i.e. you have roughly the same amount 
of different types of responses) and you are far enough down the tree to have a 
fairly pure node, it might be better to just predict at the node rather than 
guessing which direction to go. Predicting at the node will reduce the 
compounded variance of prediction observed by sending your observation further 
down the tree.
\end{itemize}

\vspace{2 mm}
\noindent
The following are reasons and situations where treating the node as terminal 
might perform worse than using a surrogate variable:

\begin{itemize}
\item If you have lopsided training data (that is you have far more 
observations of one type of response) you run the risk of always 
predicting for the response that has the most data, even when you are further 
down the tree near the terminal nodes.
\item It may perform poorly if you need to predict near the root of the tree. 
If we haven't sufficiently parsed out the data in to relatively pure nodes, 
such as near the root, we will predict essentially on what training data we 
have the most of. For example, if you could imagine having to predict on the 
root we would essentially predict which response we collected the most of.
\item If we actually have a highly correlated surrogate, then we miss out on 
possible future benefits of future splitting.
\end{itemize}

\vspace{2 mm}
\noindent
{\bf Split by Majority:} This approach assumes that an observation is most 
likely to go the direction where most of the training data goes. This method 
performs well when:

\begin{itemize}
\item Surrogate might be a bad predictor. Weak correlation might split 
erroneously.
\item If you have homogeneous data and splits are lopsided (that is, the 
majority of one type of data is going one way by a wide margin), then it is 
somewhat reasonable to assume that you would observe that observation to also 
go in that direction.
\end{itemize}

\vspace{2 mm}
\noindent
The following are reasons why splitting by majority of where the training data 
goes might perform poorly when compared to the surrogate method:

\begin{itemize}
\item Splits between two child nodes contain nearly half of the data from the 
parent. This would make for an unstable model, because assuming you would 
sample the same in another experiment, you could realize a different majority, 
thus sending your observation down the other side.
\item Again, if you have lop-sided training data, you may expect that the 
majority of the training data will always flow in favor of the response that 
makes up the majority of your training data.
\end{itemize}
%-------------------------------------------------------------------------------
%---------------------------------Problem 13------------------------------------
%-------------------------------------------------------------------------------
\newpage
\begin{center}
\framebox{\parbox{6.5in}{
{\bf{STATS 315B: Data Mining, Spring 2016}}\\
{\bf Homework 1, Due 4/28/2016}\\
{\bf Completed by: Henry Neeb, Christopher Kurrus, Tyler Chase, and Yash Vyas}
}}
\ \\
\end{center}

\section*{Problem 13}

\vspace{5 mm}
\noindent
Due to its structure, this model has an advantage over surrogate splitting in 
cases where non-response carries some sort of information.  i.e. If the primary 
split is on the Q: What is your income? In this case respondents may be 
uncomfortable answering if they are shy, so a missing value could indicate 
shyness.  This is a very elementary example, but because the model accounts for 
missing values and treats them as a variable response, effects like these will 
be accounted for.  

\vspace{5 mm}
\noindent
On the other hand, when the missing values are purely random, you would have 
additional non-optimal splits competing unnecessarily which will result in less 
data further down the tree.  This will result in watered down data which will 
run out more quickly, and is unnecessary when the additional splits are not 
improving the model prediction.  This strategy does allow for a surrogate 
effect, as if you are splitting on a predictor $Y$ and predictor say  $Z$ is 
highly correlated to $Y$, then the data that had missing values for $Y$ will 
most likely be next split on $Z$.  This can be thought of as a surrogate effect.  
This method cannot be used in cases where there is no missing data, as it will 
not create any missing value splits in the resulting model.  We can work around 
this by removing a certain percentage of the dataset at random, or 
bootstrapping our dataset and removing data from the bootstrapped points, 
including those in our original dataset.

\end{document}