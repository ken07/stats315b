\documentclass[11pt]{article}
\usepackage{subfigure}
\usepackage{color}
\usepackage{url}
\usepackage{graphicx}
\usepackage{fullpage}
\usepackage[english]{babel}
\usepackage{amssymb}
\usepackage{amsmath}
\usepackage{fancyhdr}
\usepackage{hyperref}
\usepackage{algorithmic}
\usepackage{algorithm}
\usepackage{enumerate}
\usepackage{mdframed}

\DeclareMathOperator*{\argmin}{argmin}
\newcommand*{\argminl}{\argmin\limits}


\begin{document}
\begin{center}
%-------------------------------------------------------------------------------
%---------------------------------Header----------------------------------------
%--------------------------------------------------------------------------------

\framebox{\parbox{6.5in}{
{\bf{STATS 315B: Data Mining, Spring 2016}}\\
{\bf Homework 1, Due 4/28/2016}\\
{\bf Completed by: Henry Neeb, Christopher Kurrus, Tyler Chase, and Yash Vyas}
}}
\ \\
\end{center}

%-------------------------------------------------------------------------------
%---------------------------------Answer----------------------------------------
%-------------------------------------------------------------------------------

\vspace{5 mm}
\noindent
{\bf Treat Node as Terminal:} Treating the node as terminal means we predict on 
the majority at the node with the missing value. The following are reasons and 
situations when this strategy is ok:

\begin{itemize}
\item Surrogate might be a bad predictor. Weak correlation might split 
erroneously.
\item If you start with homogenous data (i.e. you have roughly the same amount 
of different types of responses) and you are far enough down the tree to have a 
fairly pure node, it might be better to just predict at the node rather than 
guessing. Guessing at the node will reduce the compounded variance of 
prediction observed by sending your observation further down the tree.
\end{itemize}

\vspace{2 mm}
\noindent
The following are reasons and situations where this might be bad

\begin{itemize}
\item If you have lop-sided training data (that is you have far more 
observations of one type of response) you run the risk of just always 
predicting for the response that has the most data, even when you are further 
down the tree near the terminal nodes.
\item It may perform poorly if you need to predict near the root of the tree. 
If we haven't sufficiently parsed out the data in to relatively pure nodes, 
such as near the root, we will predict essentially on what training data we 
have the most of. For example, if you could imagine having to predict on the, 
root we would essentially predict which response we collected the most of.
\item If we actually have a highly correlated surrogate, then we miss out on 
possible future benefits of future splitting.
\end{itemize}

\vspace{2 mm}
\noindent
{\bf Split by Majority:} This approach assumes that an observation is most 
likely to go the direction where most of the training data goes. This method 
performs well when:

\begin{itemize}
\item Surrogate might be a bad predictor. Weak correlation might split 
erroneously.
\item If you have homogenous data and splits are lop-sided (that is, the 
majority of one type of data is going one way by a wide margin), then it is 
somewhat reasonable to assume that you would observe that oberservation to also 
go in that direction.
\end{itemize}

\vspace{2 mm}
\noindent
This method performs poorly when

\begin{itemize}
\item Splits between two child nodes contain nearly half of the data from the 
parent. This would make for an unstable model, because assuming you would 
sample the same in another experiment, you could realize a different majority, 
thus sending your observation down the other side.
\item Again, if you have lop-sided training data, you may expect that the 
majority of the training data will always flow in favor of the response that 
makes up the majority of your training data.
\end{itemize}

\end{document}