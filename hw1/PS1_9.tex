\documentclass[11pt]{article}
\usepackage{subfigure}
\usepackage{color}
\usepackage{url}
\usepackage{graphicx}
\usepackage{fullpage}
\usepackage[english]{babel}
\usepackage{amssymb}
\usepackage{amsmath}
\usepackage{fancyhdr}
\usepackage{hyperref}
\usepackage{algorithmic}
\usepackage{algorithm}
\usepackage{enumerate}
\usepackage{mdframed}

\DeclareMathOperator*{\argmin}{argmin}
\newcommand*{\argminl}{\argmin\limits}


\begin{document}
\begin{center}
%---------------------------------------------------------------------------------------
%---------------------------------Header------------------------------------------------
%---------------------------------------------------------------------------------------

\framebox{\parbox{6.5in}{
{\bf{STATS 315B: Data Mining, Spring 2016}}\\
{\bf Homework 1, Due 4/28/2016}\\
{\bf Completed by: Henry Neeb, Christopher Kurrus, Tyler Chase, and Yash Vyas}
}}
\ \\
\end{center}

%---------------------------------------------------------------------------------------
%---------------------------------Answer------------------------------------------------
%---------------------------------------------------------------------------------------

\section*{Problem 9}

\vspace{5 mm}
\noindent
To show that the minimizing values of $\hat{c}_{m}$ are given by $\sum\limits_{i=0}^N (y_{i} I(x_{i} \in R_{m})) / \sum\limits_{i=0}^N (I(x_{i} \in R_{m}))$ we will expand upon our given functions.

\vspace{3 mm}
\noindent
We know our criterion is $\sum\limits_{i=1}^N (y_{i} - F(x_{i}))^2$ and we have $F(x) = \sum\limits_{m=1}^M (c_{m}I(x \in R_{m}))$

\vspace{3 mm}
\noindent
Proceeding, we can expand our criterion into $\sum\limits_{i=1}^N (y_{i} - \sum\limits_{m=1}^M (c_{m}I(x_{i} \in R_{m})))^2$

\vspace{3 mm}
\noindent
Now our goal is to find the minimizing value $min_{c_{m}}[\sum\limits_{i=1}^N (y_{i} - \sum\limits_{m=1}^M (c_{m}I(x_{i} \in R_{m})))^2]$ 

\vspace{3 mm}
\noindent
We will find our minimizing values by taking the partial derivative on $c_{m}$ for an arbitrary region m.

\vspace{3 mm}
\noindent
$\frac{d}{dc_{m}}[\sum\limits_{i=1}^N (y_{i} - \sum\limits_{m=1}^M (c_{m}I(x_{i} \in R_{m})))^2] = 2\sum\limits_{i=1}^N [(y_{i} - \sum\limits_{m=1}^M (c_{m}I(x_{i} \in R_{m})))(I(x_{i} \in R_{m}))]$ 

\vspace{3 mm}
\noindent
Now that we have done our differentiation we simplify our terms by distributing the indicator term that is evaluating on our chosen m to the remaining terms that are non-chosen values of m. For our $y_{i}$ term the indicator stays, but this isn't true for all values in the summation for $(c_{m}I(x_{i} \in R_{m}))$.  We know that the regions are all disjoint, so every cross term between our indicator for the chosen region, and indicators for every other region will evaluate to zero.  This leaves only the $(c_{m}I(x_{i} \in R_{m}))I(x_{i} \in R_{m})$ term, but the indicator squared is merely the indicator, so we end up with the statement below.

\vspace{3 mm}
\noindent
$\sum\limits_{i=1}^N [y_{i}I(x_{i} \in R_{m}) - c_{m}I(x_{i} \in R_{m})] = 0$ 

\vspace{3 mm}
\noindent
Splitting the summation we have $\sum\limits_{i=1}^N (y_{i}I(x_{i} \in R_{m}) - c_{m}\sum\limits_{i=1}^N (I(x_{i} \in R_{m}) = 0$

\vspace{3 mm}
\noindent
So simplifying, ${c}_{m} = \frac{\sum\limits_{i=0}^N (y_{i} I(x_{i} \in R_{m}))}{\sum\limits_{i=0}^N (I(x_{i} \in R_{m}))}$ as desired.

\end{document}
