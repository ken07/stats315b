\documentclass[11pt]{article}
\usepackage{subfigure}
\usepackage{color}
\usepackage{url}
\usepackage{graphicx}
\usepackage{fullpage}
\usepackage[english]{babel}
\usepackage{amssymb}
\usepackage{amsmath}
\usepackage{fancyhdr}
\usepackage{hyperref}
\usepackage{algorithmic}
\usepackage{algorithm}
\usepackage{enumerate}
\usepackage{mdframed}
\usepackage{mathrsfs}



\begin{document}
\begin{center}
%---------------------------------------------------------------------------------------
%---------------------------------Header------------------------------------------------
%---------------------------------------------------------------------------------------

\framebox{\parbox{6.5in}{
{\bf{STATS 315B: Data Mining, Spring 2016}}\\
{\bf Homework 1, Due 4/28/2016}\\
{\bf Completed by: Henry Neeb, Christopher Kurrus, Tyler Chase, and Yash Vyas}
}}
\ \\
\end{center}
\section*{Problem 10}
\setlength{\parindent}{0pt}

From the lecture notes the improvement in square error risk when one of the regions $R_m$ is split into daughter regions $R_m \to R_l \cup R_r$

$\underset{x}{\operatorname{argmax}}$

\begin{equation}
m^* = \underset{1 \leq m \leq M}{\operatorname{argmax}} \sum_{\underline{X}_i \in R_m} \Bigg\{\Big[y_i-\bar{y}_m\Big]^2-\Big[y_i - \bar{y}_m^{(l)}I(\underline{X}_i \in R_m^{(l)})-\bar{y}_m^{(r)}I(\underline{X}_i \in R_m^{(r)})\Big]^2\Bigg\}
\end{equation}

We now expand the squared terms. 

\begin{equation}
\begin{split}
m^* = \underset{1 \leq m \leq M}{\operatorname{argmax}} \sum_{\underline{X}_i \in R_m} \Bigg\{(y_i^2-y_i^2) -2y_i\bar{y}_m + \bar{y}_m^2 +2y_i\bar{y}_m^{(l)}I(X_i \in R_m^{(l)}) +2y_i\bar{y}_m^{(r)}I(\underline{X}_i \in R_m^{(r)}) \\- 2y_m^{(l)}I(\underline{X}_i \in R_m^{(l)})\bar{y}_m^{(r)}I(\underline{X}_i \in R_m^{(r)})-[\bar{y}_m^{(l)}I(\underline{X}_i \in R_m^{(l)})]^2-[\bar{y}_m^{(r)}I(\underline{X}_i \in R_m^{(r)})]^2\Bigg\}
\end{split}
\end{equation}

Cancel the two $y_i^2$ terms. Since $R_m^{(l)} \cap R_m^{(r)} = 0$ the term  $2y_m^{(l)}I(\underline{X}_i \in R_m^{(l)})\bar{y}_m^{(r)}I(\underline{X}_i \in R_m^{(r)})$ cancels. Also note that $I(\underline{X}_i \in R)^2 = I(\underline{X}_i \in R)$.

\begin{equation}
\begin{split}
m^* = \underset{1 \leq m \leq M}{\operatorname{argmax}} \sum_{\underline{X}_i \in R_m} \Bigg\{-2y_i\bar{y}_m + \bar{y}_m^2 + 2y_i\bar{y}_m^{(l)}I(\underline{X}_i \in R_m^{(l)}) + 2y_i\bar{y}_m^{(r)}I(\underline{X}_i \in R_m^{(r)})\\ - [\bar{y}_m^{(l)}]^2I(\underline{X}_i \in R_m^{(l)}) - [\bar{y}_m^{(r)}]^2I(\underline{X}_i \in R_m^{(r)})\Bigg\}
\end{split}
\end{equation}

Next we use the fact that $\sum_{\underline{X_i} \in R_m} y_i = N_m\frac{1}{N_m} \sum_{\underline{X}_i \in R_m}y_i = N_m \bar{y}_m$

\begin{equation}
\begin{split}
m^* = \underset{1 \leq m \leq M}{\operatorname{argmax}} \Bigg\{-2N_m\bar{y}_m^2 +N_m\bar{y}_m^2 + 2N_m^{(l)}[\bar{y}_m^{(l)}]^2 + 2N_m^{(r)}[\bar{y}_m^{(r)}]^2  -N_m^{(l)}[\bar{y}_m^{(l)}]^2 -N_m^{(r)}[\bar{y}_m^{(r)}]^2\Bigg\}
\end{split}
\end{equation}

Cancel the like terms. 

\begin {equation}
\begin{split}
m^* = \underset{1 \leq m \leq M}{\operatorname{argmax}} \Bigg\{-N_m[\bar{y}_m]^2 + N_m^{(l)}[\bar{y}_m^{(l)}]^2 +N_m^{(r)}[\bar{y}_m^{(r)}]^2\Bigg\}
\end{split}
\end{equation}

We next use the fact that $N_m\bar{y}_m = N_m^{(l)}\bar{y}_m^{(l)} + N_m^{(r)}\bar{y}_m^{(r)}$ proven below.

$$ \sum_{\underline{X}_i \in R_m} y_i = \sum_{\underline{X}_i \in R_m^{(l)}}y_i + \sum_{\underline{X}_i \in R_m^{(r)}} y_i \Rightarrow N_m \frac{1}{N_m} \sum_{\underline{X}_i \in R_m}y_i = N_m^{(l)}\frac{1}{N_m^{(l)}} \sum_{\underline{X}_i \in R_m^{(l)}}y_i + N_m^{(r)}\frac{1}{N_m^{(r)}}\sum_{X_i \in R_m^{(r)}}y_i $$

\begin {equation}
m^* = \underset{1 \leq m \leq M}{\operatorname{argmax}} \Bigg\{ -N_m ( \frac{N_m^{(l)}\bar{y}_m^{(l)} + N_m^{(r)}\bar{y}_m^{(r)}}{N_m})^2 + N_m^{(l)}[\bar{y}_m^{(l)}]^2 + N_m^{(r)}[\bar{y}_m^{(r)}]^2\Bigg\}
\end{equation}

Make a common denominator and use $N_m = N_m^{(l)} + N_m^{(r)}$.

\begin {equation}
\begin{split}
m^* = \underset{1 \leq m \leq M}{\operatorname{argmax}} \Bigg\{\frac{-[N_m^{(l)}\bar{y}_m^{(l)}]^2 - [N_m^{(r)}\bar{y}_m^{(r)}]^2-2N_m^{(l)}\bar{y}_m^{(l)}N_m^{(r)}\bar{y}_m^{(r)}}{N_m}\\ + \frac{(N_m^{(l)}+ N_m^{(r)})N_m^{(l)}[\bar{y}_m^{(l)}]^2 + (N_m^{(l)}+N_m^{(r)})N_m^{(r)}[\bar{y}_m^{(r)}]^2}{N_m}\Bigg\}
\end{split}
\end{equation}

Cancel the like terms and pull out a common factor in all terms. 

\begin{equation}
m^* = \underset{1 \leq m \leq M}{\operatorname{argmax}} \Bigg\{\frac{N_m^{(l)}N_m^{(r)}}{N_m}\Big([\bar{y}_m^{(l)}]^2-2\bar{y}_m^{(l)}\bar{y}_m^{(r)} + [\bar{y}_m^{(r)}]^2\Big)\Bigg \}
\end{equation}

\begin{equation}
m^* = \underset{1 \leq m \leq M}{\operatorname{argmax}} \Bigg\{\frac{N_m^{(l)}N_m^{(r)}}{N_m}\Big(\bar{y}_m^{(l)} - \bar{y}_m^{(r)}\Big)^2\Bigg \}
\end{equation}


\end{document}