\documentclass[11pt]{article}
\usepackage{subfigure}
\usepackage{color}
\usepackage{url}
\usepackage{graphicx}
\usepackage{fullpage}
\usepackage[english]{babel}
\usepackage{amssymb}
\usepackage{amsmath}
\usepackage{fancyhdr}
\usepackage{hyperref}
\usepackage{algorithmic}
\usepackage{algorithm}
\usepackage{enumerate}
\usepackage{mdframed}
\usepackage{mathrsfs}


\begin{document}
\begin{center}
%-------------------------------------------------------------------------------
%---------------------------------Header----------------------------------------
%-------------------------------------------------------------------------------

\framebox{\parbox{6.5in}{
{\bf{STATS 315B: Data Mining, Spring 2016}}\\
{\bf Homework 3, Due 6/3/2016}\\
{\bf Completed by: Henry Neeb, Christopher Kurrus, Tyler Chase, and Yash Vyas}
}}
\ \\
\end{center}

\section*{Problem 1}

\vspace{5 mm}
\noindent
For a multi-hidden layer problem, note that for our funcitonal form, We would 
have a summation of several compositions (equal to the number of hidden layers 
compositions) of our activation function. We will try to avoid differntiating 
and writing this lengthy composition function becuase differentiating it and 
expressing it directly will be difficult to do. Instead, we will abstract the 
layers away by define a set of nodes each with directional weighted edges. We 
will find that updating each of the edge's weights will only depend on the 
edge going to and leaving that node. 

\begin{itemize}
\item $L = $ the number of hidden layers.
\item $v_{i} = $ the $i$th vertex in our network. Let there be $n$ total 
vertices.
\item $S = $ our transfer function.
\item $w_{ij} = $ the weight from node $i$ to node $j$. Refer to the set of all 
weights as $w$.
\item $w_{t} = $ our set of weights at iteration $t$.
\item $in_{j} = \sum_{i: \exists \text{ edge } i \text{ to } j} w_{ij} out_{ij}$ 
the weighted sum of all inputs coming into vertex $j$.
\item $out_{ij} = S(in_{i})$ the output from $v_{i}$ that will be propogated 
to $v_{j}$.
\item $X = $ Our training data set. There are $n$ data points each with $p$ 
features.
\item $Y = $ Our responses to our training data. Again, there are $n$ of these.
\item $F(X; w) = $ our model.
\item $F_{t}(X; w_{t}) = \sum-$ our model at iteration $t$.
\item $E_{t} = \frac{1}{2}(Y - F(X; w_{t}))^{2} $ Our total error at iteration 
$t$. We square this because we are considering squared error loss. We apply the 
$\frac{1}{2}$ for simplification when we apply a derivative. Scaling by a 
constant should not affect our outcome.
\end{itemize}

\vspace{5 mm}
\noindent
When we compute each node's share of the error and thus it's update to the 
weights, we have to first look at the nodes closest to the response/output 
node, compute their share, and then propogate it backwards. This defines a 
recurssive relationship: the nodes closest to the input layer needs to 
recurssively solve their share of the error dependent on the nodes closer to 
the output layer. We must then define our recurrsive base case (output layer) 
and our recurrsive step. We will consider the base case first

\vspace{5 mm}
\noindent
Our base case has direct access to our error at time $t$. Note that if we think 
about our prediction node from its perspective, it does not need to know what 
transpired in the hidden layers before the one directly connected to it. 
Consider the following:

\begin{gather*}
%-------------------------------line 1------------------------------------------
\text{We can rewrite our } E_{t} \text{ as } 
\frac{1}{2}(Y - F(X; w_{t}))^{2} = 
\frac{1}{2}(Y - in_{pred})^{2} = 
\frac{1}{2}(Y - 
\sum_{i: \exists \text{ edge } i \text{ to } pred} w_{i,pred} out_{i,pred})
\text{ where } pred \text{ is our prediction node}\\
%-------------------------------line 2------------------------------------------
\text{The error's gradient is then } 
\frac{\partial}{\partial w_{i,pred}} E_{t} = 
\frac{\partial}{\partial w_{i,pred}} \frac{1}{2}(Y - F(X; w_{t}))^{2}\\
%-------------------------------line 3------------------------------------------
\text{n}
%-------------------------------line 3------------------------------------------
\text{Since }
\end{gather*}

\end{document}