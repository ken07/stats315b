\documentclass[11pt]{article}
\usepackage{subfigure}
\usepackage{color}
\usepackage{url}
\usepackage{graphicx}
\usepackage{fullpage}
\usepackage[english]{babel}
\usepackage{amssymb}
\usepackage{amsmath}
\usepackage{fancyhdr}
\usepackage{hyperref}
\usepackage{algorithmic}
\usepackage{algorithm}
\usepackage{enumerate}
\usepackage{mdframed}
\usepackage{mathrsfs}


\begin{document}
\begin{center}
%---------------------------------------------------------------------------------------
%---------------------------------Header------------------------------------------------
%---------------------------------------------------------------------------------------

\framebox{\parbox{6.5in}{
{\bf{STATS 315B: Data Mining, Spring 2016}}\\
{\bf Homework 3, Due 6/03/2016}\\
{\bf Completed by: Henry Neeb, Christopher Kurrus, Tyler Chase, and Yash Vyas}
}}
\ \\
\end{center}

\section*{Problem 1}

\vspace{5 mm}
\noindent
 Consider a radial basis function network with spherical Gaussian basis of the form
 
 $$B(\underline{x}|\mu_m, \sigma_m) = exp\bigg[-\frac{1}{2\sigma_m^2}\sum_{i = 1}^{n}(x_i-\mu_{im})^2\bigg]$$
 
 With the function approximation given by 
 
 $$\hat{F}(\underline{x},\underline{\omega}) = a_0 + \sum_{m=1}^{M}a_mB(\underline{x}|\mu_m, \sigma_m)$$
 
 \begin{center}
 where $\underline{\omega}$ is all of the parameters in the model
 \end{center}
 
 Consider the residual from one observation in order to derive the gradient
 
 $$q(\underline{x},\underline{\omega}) = \frac{1}{2} \bigg[ y - \hat{F}(\underline{x},\underline{\omega})\bigg]^2$$
 
 We want to derive the gradient of the network. 
 
 \begin{equation}
 G(x) = -g_k(\underline{x}, \underline{w}) = - \frac{\partial q(\underline{x}, \underline{\omega})}{\partial\omega_k} = \bigg(- \frac{\partial q(\underline{x}, \underline{\omega})}{\partial a_m}, - \frac{\partial q(\underline{x}, \underline{\omega})}{\partial \sigma_m}, - \frac{\partial q(\underline{x}, \underline{\omega})}{\partial \mu_{im}}\bigg)
 \end{equation}
 
 \begin{equation}
 - \frac{\partial q(\underline{x}, \underline{\omega})}{\partial a_m} = \Big(y - \hat{F}(\underline{x}, \underline{\omega}) \Big)B(\underline{x}|\mu_m, \sigma_m) 
 \end{equation}
 

\begin{equation}
\begin{split}
 - \frac{\partial q(\underline{x}, \underline{\omega})}{\partial \sigma_m} & = \Big(y - \hat{F}(\underline{x}, \underline{\omega}) \Big) a_m B(\underline{x}|\mu_m, \sigma_m) \frac{1}{\sigma_m^3} \sum_{i = 1}^{n}(x_i - \mu_{im})^2\\ & = -\Big(y - \hat{F}(\underline{x}, \underline{\omega}) \Big) a_m B(\underline{x}|\mu_m, \sigma_m) \frac{2}{\sigma_m} ln \bigg[B(\underline{x}|\mu_m, \sigma_m) \bigg]
 \end{split}
 \end{equation}
 
 \begin{equation}
 - \frac{\partial q(\underline{x}, \underline{\omega})}{\partial \mu_{im}} = \Big(y - \hat{F}(\underline{x}, \underline{\omega}) \Big) a_m B(\underline{x}|\mu_m, \sigma_m) \frac{1}{\sigma_m^2}(x_i - \mu_{im})
 \end{equation}


\vspace{5 mm}
\noindent


\end{document}