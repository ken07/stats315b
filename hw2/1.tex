\documentclass[11pt]{article}
\usepackage{subfigure}
\usepackage{color}
\usepackage{url}
\usepackage{graphicx}
\usepackage{fullpage}
\usepackage[english]{babel}
\usepackage{amssymb}
\usepackage{amsmath}
\usepackage{fancyhdr}
\usepackage{hyperref}
\usepackage{algorithmic}
\usepackage{algorithm}
\usepackage{enumerate}
\usepackage{mdframed}
\usepackage{mathrsfs}


\begin{document}
\begin{center}
%---------------------------------------------------------------------------------------
%---------------------------------Header------------------------------------------------
%---------------------------------------------------------------------------------------

\framebox{\parbox{6.5in}{
{\bf{STATS 315B: Data Mining, Spring 2016}}\\
{\bf Homework 2, Due 5/22/2016}\\
{\bf Completed by: Henry Neeb, Christopher Kurrus, Tyler Chase, and Yash Vyas}
}}
\ \\
\end{center}

\section*{Problem 1}

\vspace{5 mm}
\noindent
The advantage of random variable selection strategy in random forests is that
we reduce the possibility of generating highly correlated bagged trees. Highly
correlated trees lead to low prediction accuracy. By only considering a subset 
of split variables, we force ourselves to split differently at each tree. This 
enhances prediction because if we split on all variables we tend to still split
on the same best variables, so there is little randomness and averaging the 
prediction value won't give us much of a different answer from just making
one tree. The random variable selection strategy works best if the there are 
very few interactions among the features. Another advantage of the method is 
the increased computational efficiency due to less feature exploration. 

The major disadvantage of random variable selection strategy is not being able to 
capture the interaction effects among the features in most of the bagged trees. An 
interaction between two variables is said to have been captured if they are on the same
branch of the generated tree. These interactions are not necessarily captured 
due to random variable selection for the split. Also, one does not have the liberty 
to split on the best feature due to the previous constraint. This leads to increase
in the bias of the trees generated on average and may lead to decrease in 
prediction accuracy when many trees with low predictive power are averaged.

An alternate startegy to random variable selection in the random forests is to 
randomly change the size of bootstrap sample and allow all variables to be split at 
each node. This strategy introduces variation in the trees due to sampling and 
ensures capturing interaction effects among the features. 


\vspace{5 mm}
\noindent


\end{document}