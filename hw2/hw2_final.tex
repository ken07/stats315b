\documentclass[11pt]{article}
\usepackage{subfigure}
\usepackage{color}
\usepackage{url}
\usepackage{graphicx}
\usepackage{fullpage}
\usepackage[english]{babel}
\usepackage{amssymb}
\usepackage{amsmath}
\usepackage{fancyhdr}
\usepackage{hyperref}
\usepackage{algorithmic}
\usepackage{algorithm}
\usepackage{enumerate}
\usepackage{mdframed}
\usepackage{mathrsfs}


\begin{document}
\begin{center}
%-------------------------------------------------------------------------------
%---------------------------------Problem 1-------------------------------------
%-------------------------------------------------------------------------------

\framebox{\parbox{6.5in}{
{\bf{STATS 315B: Data Mining, Spring 2016}}\\
{\bf Homework 2, Due 5/22/2016}\\
{\bf Completed by: Henry Neeb, Christopher Kurrus, Tyler Chase, and Yash Vyas}
}}
\ \\
\end{center}

\section*{Problem 1}

\vspace{5 mm}
\noindent
The advantage of random variable selection strategy in random forests is that
we reduce the possibility of generating highly correlated bagged trees. Highly
correlated trees lead to low prediction accuracy. By only considering a subset 
of split variables, we force ourselves to split differently at each tree. This 
enhances prediction because if we split on all variables we tend to still split
on the same best variables, so there is little randomness and averaging the 
prediction value won't give us much of a different answer from just making
one tree. The random variable selection strategy works best if the there are 
very few interactions among the features. Another advantage of the method is 
the increased computational efficiency due to less feature exploration. 

\vspace{5 mm}
\noindent
The major disadvantage of random variable selection strategy is not being able 
to capture the interaction effects among the features in most of the bagged 
trees. An interaction between two variables is said to have been captured if 
they are on the samebranch of the generated tree. These interactions are not 
necessarily captured due to random variable selection for the split. Also, one 
does not have the liberty to split on the best feature due to the previous 
constraint. This leads to increasein the bias of the trees generated on average 
and may lead to decrease in prediction accuracy when many trees with low 
predictive power are averaged.

\vspace{5 mm}
\noindent
An alternate startegy to random variable selection in the random forests is to 
randomly change the size of bootstrap sample and allow all variables to be 
split at each node. This strategy introduces variation in the trees due to 
sampling and ensures capturing interaction effects among the features. 

%-------------------------------------------------------------------------------
%---------------------------------Problem 2-------------------------------------
%-------------------------------------------------------------------------------
\begin{center}
\framebox{\parbox{6.5in}{
{\bf{STATS 315B: Data Mining, Spring 2016}}\\
{\bf Homework 2, Due 5/22/2016}\\
{\bf Completed by: Henry Neeb, Christopher Kurrus, Tyler Chase, and Yash Vyas}
}}
\ \\
\end{center}

\section*{Problem 2}

\vspace{5 mm}
\noindent
Linear regression uses least squares as the error function. When the number of 
predictors is greater than the number of observations in the training data, we 
have multiple perfect fits for the model and therefore, infinite solutions for 
the coefficient estimates of the least squares. Thus, the model would have a 
huge variance.

\vspace{5 mm}
\noindent
Regularization is necessary to find the unique coefficient estimates when the 
number of observations is less than the number of predictors. If there is high 
collinearity/correlation among the predictors in the model, then a 
regularization such as ridge penalty reduces the variation in coefficient 
estimates of the variables. Another case where regularization helps is when 
there exist multiple noisy variables in the data. In such a case, 
regularization such as a lasso penalty ensures that the coefficients of noisy 
variables are zero.

\vspace{5 mm}
\noindent
Regularization is disadvantageous when the number of predictors is extremely 
small and predictors are believed to be  independent. In this case, we may want 
to pursue feature augmentation by incorporating the higher orders and 
interaction among parameters. But the idea of regularization goes against 
feature augmentation. Another example where regularization fails is when we 
have a perfect line/plane that fits the model for linear regression. Generally, 
regularization fails when the bias introduced by it does not lead to drastic 
decrease in the variance of prediction. 

\vspace{5 mm}
\noindent
During boosting, we generate a linear combination of all possible trees. In 
this situation sparsity is assumed because out of all possible trees there are 
only a few trees that capture the signal in the data. 

\vspace{5 mm}
\noindent
Most of the trees generated in boosting do not influence the outcome variable. 
If we assume sparsity and it exists then our predictions are good; otherwise, 
our predictions are inaccurate. However, if we do not assume sparsity and 
our variable space is larger than our sample space we are stuck. We simply do 
not have enough data to estimate a large number of parameters (all possible 
trees in the boosting context). Hence, the assumption of sparsity is always 
reasonable in boosting.

%-------------------------------------------------------------------------------
%---------------------------------Problem 3-------------------------------------
%-------------------------------------------------------------------------------

\end{document}