\documentclass[11pt]{article}
\usepackage{subfigure}
\usepackage{color}
\usepackage{url}
\usepackage{graphicx}
\usepackage{fullpage}
\usepackage[english]{babel}
\usepackage{amssymb}
\usepackage{amsmath}
\usepackage{fancyhdr}
\usepackage{hyperref}
\usepackage{algorithmic}
\usepackage{algorithm}
\usepackage{enumerate}
\usepackage{mdframed}
\usepackage{mathrsfs}


\begin{document}
\begin{center}
%---------------------------------------------------------------------------------------
%---------------------------------Header------------------------------------------------
%---------------------------------------------------------------------------------------

\framebox{\parbox{6.5in}{
{\bf{STATS 315B: Data Mining, Spring 2016}}\\
{\bf Homework 2, Due 5/22/2016}\\
{\bf Completed by: Henry Neeb, Christopher Kurrus, Tyler Chase, and Yash Vyas}
}}
\ \\
\end{center}

\section*{Problem 2}

\vspace{5 mm}
\noindent
Linear regression uses least squares as the error function. When the number of 
predictors is greater than the number of observations in the training data, we 
have multiple perfect fits for the model and therefore, infinite solutions for 
the coefficient estimates of the least squares. Thus, the model would have a 
huge variance.

\vspace{5 mm}
\noindent
Regularization is necessary to find the unique coefficient estimates when the 
number of observations is less than the number of predictors. If there is high 
collinearity/correlation among the predictors in the model, then a 
regularization such as ridge penalty reduces the variation in coefficient 
estimates of the variables. Another case where regularization helps is when 
there exist multiple noisy variables in the data. In such a case, 
regularization such as a lasso penalty ensures that the coefficients of noisy 
variables are zero.

\vspace{5 mm}
\noindent
Regularization is disadvantageous when the number of predictors is extremely 
small and predictors are believed to be  independent. In this case, we may want 
to pursue feature augmentation by incorporating the higher orders and 
interaction among parameters. But the idea of regularization goes against 
feature augmentation. Another example where regularization fails is when we 
have a perfect line/plane that fits the model for linear regression. Generally, 
regularization fails when the bias introduced by it does not lead to drastic 
decrease in the variance of prediction. 

\vspace{5 mm}
\noindent
During boosting, we generate a linear combination of all possible trees. In 
this situation sparsity is assumed because out of all possible trees there are 
only a few trees that capture the signal in the data. 

\vspace{5 mm}
\noindent
Most of the trees generated in boosting do not influence the outcome variable. 
If we assume sparsity and it exists then our predictions are good; otherwise, 
our predictions are inaccurate. However, if we do not assume sparsity and it 
does not exist even then our predictions are inaccurate.  Hence, the assumption 
of sparsity is always reasonable in boosting.


%We simply do not have enough data to estimate a large number of parameters (all possible trees in the boosting context). 

\vspace{5 mm}
\noindent


\end{document}