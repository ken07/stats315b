\documentclass[11pt]{article}
\usepackage{subfigure}
\usepackage{color}
\usepackage{url}
\usepackage{graphicx}
\usepackage{fullpage}
\usepackage[english]{babel}
\usepackage{amssymb}
\usepackage{amsmath}
\usepackage{fancyhdr}
\usepackage{hyperref}
\usepackage{algorithmic}
\usepackage{algorithm}
\usepackage{enumerate}
\usepackage{mdframed}


\begin{document}
\begin{center}
%-------------------------------------------------------------------------------
%---------------------------------Header----------------------------------------
%-------------------------------------------------------------------------------

\framebox{\parbox{6.5in}{
{\bf{STATS 315B: Data Mining, Spring 2016}}\\
{\bf Homework 2, Due 5/22/2016}\\
{\bf Completed by: Henry Neeb, Christopher Kurrus, Tyler Chase, and Yash Vyas}
}}
\ \\
\end{center}

%-------------------------------------------------------------------------------
%---------------------------------Answer----------------------------------------
%-------------------------------------------------------------------------------

\section*{Problem 3}

\noindent
For convex empirical risk and convex penalities, the solution of 
$\hat{a}(\lambda)$ will be unique since it is a sum of convex functions. We 
will now attempt to minimize the sum of the empircal risk and the penalty. 
First, we consider minimizing the empirical risk:

%-------------------------------------------------------------------------------
%------------------------Emperical Risk Derivative------------------------------
%-------------------------------------------------------------------------------
\begin{gather*}
%------------------------------line 1-------------------------------------------
\text{Let } L(y_{i}, a_{0} + \sum_{j = 1}^{n} a_{j}x_{ij}) = L 
\text{ and } a_{0} + \sum_{j = 1}^{n} a_{j}x_{ij} = f(x, a) = f\\
%------------------------------line 2-------------------------------------------
\frac{\partial \hat{R}}{\partial a_{k}} = 
\frac{1}{N} \sum_{j = 1}^{N} \frac{\partial L}{\partial a_{k}}\\
%------------------------------line 3-------------------------------------------
\text{By the chain rule: } 
\frac{\partial L}{\partial a_{k}} = 
\frac{\partial L(y_{i}, f(x, a))}{\partial a_{k}} = 
\frac{\partial L(y_{i}, f(x, a))}{f(x, a)} 
\frac{\partial f(x, a)}{\partial a_{k}} = 
\frac{\partial L}{\partial f} \frac{\partial f}{\partial a_{k}}\\
%------------------------------line 4-------------------------------------------
\text{Now consider } 
\frac{\partial f}{\partial a_{k}} = 
\frac{\partial}{\partial a_{k}} [a_{0} + \sum_{j = 1}^{n} a_{j}x_{ij}]\\
%------------------------------line 5-------------------------------------------
\text{Note that } a_{k} 
\text{ only appears in the sumation and is multiplied by } x_{ik}\\
%------------------------------line 6-------------------------------------------
\text{So all values in } f 
\text{ will evaluate to zero except } a_{k}x_{ik}\\
%------------------------------line 7-------------------------------------------
\text{Then } 
\frac{\partial}{\partial a_{k}} [a_{0} + \sum_{j = 1}^{n} a_{j}x_{ij}] = x_{ik}\\
%------------------------------line 8-------------------------------------------
\text{And } 
\frac{1}{N} \sum_{i = 1}^{N} \frac{\partial L}{\partial a_{k}} = 
\frac{1}{N} \sum_{i = 1}^{N} \frac{\partial L}{\partial f} 
\frac{\partial f}{\partial a_{k}} = 
\frac{1}{N} \sum_{i = 1}^{N} \frac{\partial L}{\partial f} x_{ik}
\end{gather*}

\vspace{5mm}
\noindent
Now we consider the penalty function for $\gamma \ge 1$:

%-------------------------------------------------------------------------------
%---------------------------Penalty Derivative----------------------------------
%-------------------------------------------------------------------------------
\begin{gather*}
%------------------------------line 1-------------------------------------------
\frac{\partial P_{\gamma} (a)}{\partial a_{k}} = 
\frac{\partial}{\partial a_{k}} \sum_{j = 1}^{n} |a_{j}|^{\gamma}\\
%------------------------------line 2-------------------------------------------
\text{Note, the only value in this sum that depends on } a_{k} 
\text{ is } |a_{k}|^{\gamma}\\
%------------------------------line 3-------------------------------------------
\text{So, } \frac{\partial}{\partial a_{k}} \sum_{j = 1}^{n} |a_{j}|^{\gamma} = 
\frac{\partial}{\partial a_{k}} |a_{k}|^{\gamma} = 
\gamma |a_{k}|^{\gamma - 1} \frac{\partial |a_{k}|}{\partial a_{k}} 
\text{, where } \frac{\partial |a_{k}|}{\partial a_{k}} 
\text{ is the subgradient and } 
\frac{\partial |a_{k}|}{\partial a_{k}} \in [-1, 1]\\
\end{gather*}

\newpage

\noindent
Note that the above holds for $\gamma = 1$. Finally, together we have:

%-------------------------------------------------------------------------------
%---------------------------Combined Derivative---------------------------------
%-------------------------------------------------------------------------------
\begin{gather*}
\frac{\partial}{\partial a_{k}} [\hat{R}(a) + \lambda P_{\gamma}(a)] = 
\frac{1}{N} \sum_{i = 1}^{N} \frac{\partial L}{\partial f} x_{ik} + 
\gamma \lambda |a_{k}|^{\gamma - 1} \frac{\partial |a_{k}|}{a_{k}}
\end{gather*}

\vspace{5mm}
\noindent
To find the minimum, we would need to set the above equal to zero and find the 
values of $a$ that satisfy the equality. Now, let's assume that $\gamma > 1$. 
We will show by contradiction that values of $a$ are not forced to zero. Assume 
that some $a_{k} = 0$, then:

%-------------------------------------------------------------------------------
%---------------------------Proof by Contradiction------------------------------
%-------------------------------------------------------------------------------
\begin{gather*}
%------------------------------line 1-------------------------------------------
\frac{1}{N} \sum_{i = 1}^{N} \frac{\partial L}{\partial f} x_{ik} + 
\gamma |a_{k}|^{\gamma - 1} \frac{\partial |a_{k}|}{a_{k}} = 
\frac{1}{N} \sum_{i = 1}^{N} \frac{\partial L}{\partial f} x_{ik} + 
\gamma \lambda \times 0^{\gamma - 1} \times \frac{\partial |a_{k}|}{a_{k}} = 
%------------------------------line 2-------------------------------------------
\frac{1}{N} \sum_{i = 1}^{N} \frac{\partial L}{\partial f} x_{ik}\\
\text{Then } 
\frac{\partial}{\partial a_{k}} [\hat{R}(a) + \lambda P_{\gamma}(a)] = 0 \implies 
\frac{1}{N} \sum_{i = 1}^{N} \frac{\partial L}{\partial f} x_{ik} = 0
\end{gather*}

\vspace{5mm}
\noindent
Note that the above is just the derivative of the emperical risk set equal to 
zero. Its solution then will just be the standard unpenalized solution and 
will not necessarily equal zero. This is a contradiction to our assumption that 
$a_{k} = 0$, thus $a_{k} \ne 0$ and all coefficients along the path indexed by 
$\lambda$ are not necessarily equal to zero.

\vspace{5mm}
\noindent
We will now show that getting zero coefficients is possible for the lasso 
penalty. Assume $\gamma = 1$.

%-------------------------------------------------------------------------------
%-------------------------------Lasso Proof-------------------------------------
%-------------------------------------------------------------------------------
\begin{gather*}
%------------------------------line 1-------------------------------------------
\frac{1}{N} \sum_{i = 1}^{N} \frac{\partial L}{\partial f} x_{ik} + 
\gamma \lambda |a_{k}|^{\gamma - 1} \frac{\partial |a_{k}|}{a_{k}} = 
\frac{1}{N} \sum_{i = 1}^{N} \frac{\partial L}{\partial f} x_{ik} + 
\lambda \frac{\partial |a_{k}|}{a_{k}}\\
%------------------------------line 2-------------------------------------------
\text{For } a_{k} \text{ near zero } 
-1 \le \frac{\partial |a_{k}|}{a_{k}} \le 1 \implies 
-\lambda \le -\lambda \frac{\partial |a_{k}|}{a_{k}} \le \lambda\\
%------------------------------line 3-------------------------------------------
\text{So } 
\frac{\partial}{\partial a_{k}} [\hat{R}(a) + \lambda P_{\gamma}(a)] = 0 \implies 
-\lambda \frac{\partial |a_{k}|}{a_{k}} = 
\frac{1}{N} \sum_{i = 1}^{N} \frac{\partial L}{\partial f} x_{ik} \implies 
-\lambda \le \frac{1}{N} \sum_{i = 1}^{N} \frac{\partial L}{\partial f} x_{ik} \le \lambda
\end{gather*}

\vspace{5mm}
\noindent
So, for $a_{k}$ near zero, we can bound the solution to the derivative of the 
unpenalized loss between plus and minus $\lambda$. This means that the solution 
will be equal to zero between those values of $\lambda$, so we have shown that 
$a_{k}$ can be zero with the lasso penalty.

\vspace{5mm}
\noindent
We will now show that the elastic net has the ability to set some coefficients 
to zero.

%-------------------------------------------------------------------------------
%-------------------Elastic Net Penalty Derivative------------------------------
%-------------------------------------------------------------------------------
\begin{gather*}
%------------------------------line 1-------------------------------------------
\frac{\partial P_{\gamma}(a)}{\partial a_{k}} = 
\frac{\partial}{\partial a_{k}}[
\sum_{j = 1}^{n} (\gamma - 1) \frac{a_{j}^{2}}{2} + 
(2 - \gamma) |a_{j}|
]\\
%------------------------------line 2-------------------------------------------
\text{Again, the only values that will not evaluate to zero are the } 
a_{k} 
\text{ values. So the above evaluates to:}\\
%------------------------------line 3-------------------------------------------
(\gamma - 1) a_{k} + (2 - \gamma) \frac{\partial |a_{k}|}{a_{k}}
\end{gather*}

\vspace{5mm}
\noindent
We have shown what will happen for $\gamma = 1$ (Lasso penalty) and for 
$\gamma = 2$ (Ridge penalty). We only need to consider $1 < \gamma < 2$. With 
these constraints, we are interested in solving:

%-------------------------------------------------------------------------------
%-------------------Proof Elastic Net Forces Zero-------------------------------
%-------------------------------------------------------------------------------
\begin{gather*}
%------------------------------line 1-------------------------------------------
\frac{\partial}{\partial a_{k}} [\hat{R}(a) + \lambda P_{\gamma}(a)] = 
\frac{1}{N} \sum_{i = 1}^{N} \frac{\partial L}{\partial f} x_{ik} + 
\lambda (\gamma - 1) a_{k} + 
\lambda (2 - \gamma) \frac{\partial |a_{k}|}{a_{k}}\\
%------------------------------line 2-------------------------------------------
\text{For } a_{k} \text{ near zero } 
-1 \le \frac{\partial |a_{k}|}{a_{k}} \le 1 \implies 
-\lambda (2 - \gamma) \le 
-\lambda (2 - \gamma) \frac{\partial |a_{k}|}{a_{k}} \le 
\lambda (2 - \gamma)\\
%------------------------------line 3-------------------------------------------
\text{Then } 
\frac{1}{N} \sum_{i = 1}^{N} \frac{\partial L}{\partial f} x_{ik} + 
\lambda (\gamma - 1) a_{k} + 
\lambda (2 - \gamma) \frac{\partial |a_{k}|}{a_{k}} = 0 \implies\\
%------------------------------line 4-------------------------------------------
\implies
\frac{1}{N} \sum_{i = 1}^{N} \frac{\partial L}{\partial f} x_{ik} + 
\lambda (\gamma - 1) a_{k} = 
- \lambda (2 - \gamma) \frac{\partial |a_{k}|}{a_{k}} \implies\\
%------------------------------line 5-------------------------------------------
\implies
-\lambda (2 - \gamma) \le 
\frac{1}{N} \sum_{i = 1}^{N} \frac{\partial L}{\partial f} x_{ik} + 
\lambda (\gamma - 1) a_{k} \le
\lambda (2 - \gamma)
\end{gather*}

\vspace{5mm}
\noindent
The bounded value is a ridge-like solution (with 
$\lambda ' = (\gamma - 1) \lambda$). $a_{k}$ solutions to this near zero can 
be bounded by $\lambda (2 - \gamma)$, which implies that some coefficients can 
be forced to zero (by the same logic as why the lasso forces some coefficients 
to zero).

\end{document}